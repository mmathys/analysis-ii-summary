\subsection{Differenzierbarkeit}

\begin{Definition}{Differenzierbarkeit}{}
    Sei $\Omega \subset \R$ offen, $f$ : $\Omega \to \R$, $x_0\in\Omega$. $f$ heisst \textbf{differenzierbar an Stelle $x_0$}, falls der Grenzwert
    \[
    \lim_{x\ to x_0} \frac{f(x) - f(x_0)}{x - x_0} =:
    f'(x_0) =:
    \frac{df}{dx}
    \]
    existiert. Wir nennen $f'(x_0)$ die Ableitung (das \textbf{Differential}) von $f$ an der Stelle $x_0$. Eine solche Funktion heisst dann \textbf{differenzierbar auf $\Omega$}, wenn sie an jeder Stelle $x_0 \in \Omega$ differenzierbar ist.
\end{Definition}
$f$ diffbar $\Leftrightarrow$ alle Teil-$f$ sind diffbar.

$f,g$ diffbar $\Rightarrow$ $f+g$, $f \cdot g$, $\frac{f}{g}$, $g \circ f$ diffbar
\begin{Satz}{Kettenregel}{}
	Seien $X,Y \subset \R^n$ offen und $f: X \to Y$, $g: Y \to \R^p$ differenzierbar. Dann ist $g \circ f$ differenzierbar und die Ableitung ist \[d(g\circ f)(x_0) = dg(f(x_0))\circ df(x_0)\] und die Jakobi-Matrix ist \[J_{g\circ f}(x_0) = J_g(f(x_0))J_f(x_0)\]
\end{Satz}
\begin{Definition}{Partielle Differenzierbarkeit}{}
	$f$: $\R^n \rightarrow \R^m$, falls:

	\[
    	\lim_{h \rightarrow 0} \frac{f(x_0 + h e_i)-f(x_0)}{h} =: \frac{\partial f}{\partial x_i}(x_0)
	\]

	oder generell für alle Einheitsvektoren $e_i$ zusammengefasst in Richtung $v \in \R^n$

	\[
    	\lim_{h \rightarrow 0} \frac{f(x_0 + h v)-f(x_0)}{h} =: D_v f(x_0)
	\]

	Dieser $\lim$ existiert $\Leftrightarrow$ in Richtung $e_i$ an Stelle $x_0$ partiell differenzierbar.
\end{Definition}
\begin{Definition}{Totale Differenzierbarkeit}{}
	Sei $f: \Omega \subset \R^m \to \R^n$. $f$ heisst differenzierbar an der Stelle $x_0 \in \Omega$, falls eine lineare Abbildung $A: \R^n \to \R^m$ existiert (also eine $m \times n$ Matrix), sodass \[\lim_{x \to x_0} \frac{|f(x) - f(x_0) - A(x - x_0)|}{|x - x_0|} = 0\] Dann heisst $df(x_0) := A$ das Differenzial von $f$ in Punkt $x_0$. \\
	Diese Matrix $A = Df(x_0)$ ist gegeben durch
	\[
		\text{D}f(x_0) =
        \begin{pmatrix}
            \frac{\partial f_1}{\partial x_1}(x_0)&\hdots&\frac{\partial f_1}{\partial x_n}(x_0)\\
            \vdots&\ddots&\vdots\\
            \frac{\partial f_m}{\partial x_1}(x_0)&\hdots&\frac{\partial f_n}{\partial x_m}(x_0)
        \end{pmatrix}
    \]
\end{Definition}

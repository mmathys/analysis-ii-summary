%TODO study lagrange

\subsection{Lagrange Multiplikatoren}
Sei $f(x) \in \mathbb{R}^{n}$ die zu maximierende Funktion, von der wir aber nur Punkte
betrachten wollen, für welche gilt, dass $g(x) = 0$ mit $g(x) \in \mathbb{R}^{l}$
Definition der \textbf{Lagrange-Funktion}:
\[ L = f-\lambda \cdot g = f - \lambda_{1} g_{1} - \ldots - \lambda_{n}g_{n} ~ \text{ mit $\lambda$ in } \mathbb{R}^{l} \]
Dieses $\lambda$ existiert immer, wenn $f, g \in C^{1}$.
Die Kandidaten für Extrema von $f$ unter der Nebenbedingung $g = 0$ sind genau die
kritischen Punkte der Lagrange-Funktion $L$. Mittels der Hesse-Matrix von $L$
kann die Art der Extrema gefunden werden. Jeder Kandidat wird in $f$ eingesetzt
um zu erkennen, wo $f$ mit welchen Werten Extrema annimmt.

TODO: Rezepte zum Finden

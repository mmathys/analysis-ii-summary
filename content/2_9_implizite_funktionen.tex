%TODO study implizite funktionen

\subsection{Satz der Impliziten Funktion}

theorem 3.10.4\\

Ziel: Existenz von lokalen Umkehrungen. ''Implizit'', weil die Form der Gleichung stets ein Gleichungssystem impliziert. Form:

\[
    f(x, y) = 0
\]

$x = (x_1, ..., x_k)$, $y = (y_1, ..., y_l)$

\begin{tiny}
\[
    df(x, y) =
        \begin{pmatrix}
            \frac{\partial f_1}{\partial x_1} & \hdots & \frac{\partial f_1}{\partial x_k}
            & \frac{\partial f_1}{\partial y_1} & \hdots & \frac{\partial f_1}{\partial y_l}\\
            
            \vdots & \ddots & \vdots & \vdots & \ddots & \vdots\\
            
            \frac{\partial f_l}{\partial x_1} & \hdots & \frac{\partial f_l}{\partial x_k}
            & \frac{\partial f_l}{\partial y_1} & \hdots & \frac{\partial f_l}{\partial y_l}\\
        \end{pmatrix} \text{ wobei }
    d_yf(x, y) =
        \begin{pmatrix}
            \frac{\partial f_1}{\partial y_1} & \hdots & \frac{\partial f_1}{\partial y_l}\\
            
            \vdots & \ddots & \vdots\\
            
            \frac{\partial f_l}{\partial y_1} & \hdots & \frac{\partial f_l}{\partial y_l}\\
        \end{pmatrix}
\]
\end{tiny}

\textbf{Satz} Sei $f: \mathbb{R}^k \times \mathbb{R}^l \to \mathbb{R}^l$ stetig differenzierbar. Falls ein Punkt $p_0 = (a, b)$ regulär und dass

\[
    f(p_0) = 0 \text{ und } \det(d_y f(p_0)) \neq 0\ (\iff d_y f(p_0) \text{ ist invertierbar})
\]

dann kann man lokal nach $b$ auflösen. Äquivalent: $\iff \exists h: f(x, h(x)) = 0$\\

\textbf{Rezept: Ableitung der Auflösungsfunktion $h$:} Bsp 2-dimensional (auflösen nach $y$) oder 3-dimensional (auflösen nach $z$)

\[
    dh(x) = -\frac{d_x f(x, h(x))}{d_y f(x, h(x))}
\]


\[
    \nabla h(x, y) =
        \left(
            -\frac{\partial_x f(x, y, h(x, y))}{\partial_z f(x, y, h(x, y))},
            -\frac{\partial_y f(x, y, h(x, y))}{\partial_z f(x, y, h(x, y))}
        \right)
\]

\textbf{Rezept: Typische Aufgabe} 1. $p_0$ bestimmen, sodass $f(x, y)=0$ stimmt. 2. $df(x, y)$ berechnen und $d_y$ kontrollieren, ob $\neq 0$ 3. Satz bewiesen, Ableiten der Auflösungsfunktion.

\subsection{Satz der Impliziten Funktion}

Ziel: Existenz von lokalen Umkehrungen. ''Implizit'', weil die Form der Gleichung stets ein Gleichungssystem impliziert. Sei das Gleichungssystem der Form:

\[
    f(x, y) = 0
\]

$x = (x_1, ..., x_k)$, $y = (y_1, ..., y_l)$

\[
    df(x, y) = (df_x(x, y) \mid df_y(x, y)) = 
        \begin{pmatrix}
            \frac{\partial f_1}{\partial x_1} & \hdots & \frac{\partial f_1}{\partial x_k}
            & \frac{\partial f_1}{\partial y_1} & \hdots & \frac{\partial f_1}{\partial y_l}\\
            
            \vdots & \ddots & \vdots & \vdots & \ddots & \vdots\\
            
            \frac{\partial f_l}{\partial x_1} & \hdots & \frac{\partial f_l}{\partial x_k}
            & \frac{\partial f_l}{\partial y_1} & \hdots & \frac{\partial f_l}{\partial y_l}\\
        \end{pmatrix}
\]

\begin{Satz}{Implizite Funktionen}{}
	Sei $\Omega \subset \R^n = \R^k \times \R^l$ offen und sei $f: \Omega \to \R^l$ stetig differenzierbar. Ist der Punkt $p_0 = (a, b) \in \Omega$ (mit $a=$ erste $k$ Koordinaten und $b=$ letzte $l$ Koordinaten von $p_0$) regulär mit
	\[
    	f(p_0) = 0 \text{ und } \det(d_y f(p_0)) \neq 0\ (\iff d_y f(p_0) \text{ ist invertierbar})
	\]
	wobei $d_yf(p_0)$ die partiellen Ableitungen der Koordinaten $y_1, ..., y_l$ enthält, so lässt sich das Gleichungssystem $f(x, y) = 0$ nach Koordinaten $y$ auflösen. Das heisst es existiert ein $h: U \to V$ sodass 
	\[
  		\exists h: f(x, h(x)) = 0
	\]
\end{Satz}

\begin{Beispiel}{Zeigen, dass implizite Funktion existiert $f \rightarrow \mathbb{R}$}{}
\[ F(x, y) = x^4 + x^3y^2 - y + y^2 + y^3 - 1\]
Zeigen Sie, dass offene Intervalle $U, V \subset \mathbb{R}$ existieren, so dass $1 \in U$, $-1 \in V$
und eine $C^1$-Abbildung $f: V \rightarrow U$ mit $f(-1) = 1$ existiert, s.d. für alle $(x,y) \in U \times V$ gilt:
\[ F(x, y) = 0 \; \Longleftrightarrow x = f(x) \]
Berechnen Sie $f'(-1)$\\
\\
\textbf{Lösung}: Es gilt
\[ \partial_y F(x,y) = 2x^3y - 1 + 2y + 3y^2 ~~~ \text{ mit } ~~~ \partial_y F(-1, 1) = 2 \neq 0 \]
Dann existiert dank dem Satz über implizite Funktionen eine Funktion $f: V \rightarrow U$ mit den vorherigen
Eigenschaften und es gilt
\[ f'(1) = - \frac{\partial_x f(-1, 1)}{\partial_y f(-1, 1)} = \frac{1}{2} \]
sowie
\[ x^4 + x^3f(x)^2 - f(x) + f(x)^2 + f(x)^3 = 1 \]
\[ 4x^3 + 3x^2 f(x)^2 + 2x^3 f'(x) f(x) - f'(x) + 2f'(x)f(x) + 3f'(x)f(x)^2 = 0 \]
Da $f(-1) = 1$ gilt folglich:
\[ 0 = -4 + 3 - 2f'(-1) - f'(-1) + 2f'(-1) + 3f'(-1) = -1 + 2f'(-1)\]
also $f'(-1) = \frac{1}{2}$
\end{Beispiel}

\begin{Beispiel}{Zeigen, dass implizite Funktion existiert $f \rightarrow \mathbb{R}^2$}{}
\[ F(x,y,z) = x + y + z + \sin(xyz) \]
Zeigen Sie, dass ein offener Bereich $U \subset \mathbb{R}^2$ und ein offenes Intervall
$V \subset \mathbb{R}$ existieren, so dass $(0,0) \in U$, $U \in V$ und eine
$C^1$-Abbildung $f: U \leftarrow V$ mit $f(0,0) = 0$ existiert, sodass für alle $(x,y,z) \in U \times V$ gilt:
\[ F(x,y,z) = 0 \; \Longleftrightarrow z = f(x,y) \]
Berechnen Sie $\nabla f(0,0)$.\\
\\
\textbf{Lösung:} Es gilt
\[ \partial_z F(x,y,z) = 1 + xy \cos(xyz) ~~~ \text{ mit } ~~~ \partial_x F(0,0,0) = 1 \neq 0 \]
Daher existiert gem. Satz über implizite Funktionen die Funktion $f$ und es gilt:
\[ \nabla f(0,0) = \left( - \frac{\partial_x F(0,0,0)}{\partial_z F(0,0,0)}, ~~ -\frac{\partial_y F(0,0,0)}{\partial_z F(0,0,0)} \right) = (-1, -1) \]
\end{Beispiel} 


\begin{Rezept}{Ableitung der Auflösungsfunktion $h$:}{}
Bsp 2-dimensional (auflösen nach $y$) oder 3-dimensional (auflösen nach $z$)
\[
    dh(x) = -\frac{d_x f(x, h(x))}{d_y f(x, h(x))}
\]

\[
    \nabla h(x, y) =
        \left(
            -\frac{\partial_x f(x, y, h(x, y))}{\partial_z f(x, y, h(x, y))},
            -\frac{\partial_y f(x, y, h(x, y))}{\partial_z f(x, y, h(x, y))}
        \right)
\]
\end{Rezept}

\begin{Rezept}{Typische Aufgabe}{}
1. $p_0$ bestimmen, sodass $f(x, y)=0$ stimmt.\\
2. $df(x, y)$ berechnen und $d_y$ kontrollieren, ob $\neq 0$\\
3. Satz bewiesen, Ableiten der Auflösungsfunktion.
\end{Rezept}

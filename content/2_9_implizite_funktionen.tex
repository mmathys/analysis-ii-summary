%TODO study implizite funktionen

\subsection{Satz der Impliziten Funktion}

Ziel: Existenz von lokalen Umkehrungen. ''Implizit'', weil die Form der Gleichung stets ein Gleichungssystem impliziert. Sei das Gleichungssystem der Form:

\[
    f(x, y) = 0
\]

$x = (x_1, ..., x_k)$, $y = (y_1, ..., y_l)$

\[
    df(x, y) = (df_x(x, y) \mid df_y(x, y)) = 
        \begin{pmatrix}
            \frac{\partial f_1}{\partial x_1} & \hdots & \frac{\partial f_1}{\partial x_k}
            & \frac{\partial f_1}{\partial y_1} & \hdots & \frac{\partial f_1}{\partial y_l}\\
            
            \vdots & \ddots & \vdots & \vdots & \ddots & \vdots\\
            
            \frac{\partial f_l}{\partial x_1} & \hdots & \frac{\partial f_l}{\partial x_k}
            & \frac{\partial f_l}{\partial y_1} & \hdots & \frac{\partial f_l}{\partial y_l}\\
        \end{pmatrix}
\]

\begin{Satz}{Implizite Funktionen}{}
	Sei $\Omega \subset \R^n = \R^k \times \R^l$ offen und sei $f: \Omega \to \mathbb{R}^l$ stetig differenzierbar. Ist der Punkt $p_0 = (a, b) \in \Omega$ (mit $a=$ erste $k$ Koordinaten und $b=$ letzte $l$ Koordinaten von $p_0$) regulär mit
	\[
    	f(p_0) = 0 \text{ und } \det(d_y f(p_0)) \neq 0\ (\iff d_y f(p_0) \text{ ist invertierbar})
	\]
	wobei $d_yf(p_0)$ die partiellen Ableitungen der Koordinaten $y_1, ..., y_l$ enthält, so lässt sich das Gleichungssystem $f(x, y) = 0$ nach Koordinaten $y$ auflösen. Das heisst es existiert ein $h: U \to V$ sodass 
	\[
  		\exists h: f(x, h(x)) = 0
	\]
\end{Satz}

\textbf{Rezept: Ableitung der Auflösungsfunktion $h$:} Bsp 2-dimensional (auflösen nach $y$) oder 3-dimensional (auflösen nach $z$)

\[
    dh(x) = -\frac{d_x f(x, h(x))}{d_y f(x, h(x))}
\]


\[
    \nabla h(x, y) =
        \left(
            -\frac{\partial_x f(x, y, h(x, y))}{\partial_z f(x, y, h(x, y))},
            -\frac{\partial_y f(x, y, h(x, y))}{\partial_z f(x, y, h(x, y))}
        \right)
\]

\textbf{Rezept: Typische Aufgabe} 1. $p_0$ bestimmen, sodass $f(x, y)=0$ stimmt. 2. $df(x, y)$ berechnen und $d_y$ kontrollieren, ob $\neq 0$ 3. Satz bewiesen, Ableiten der Auflösungsfunktion.
